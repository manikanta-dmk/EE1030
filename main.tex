%iffalse
\let\negmedspace\undefined
\let\negthickspace\undefined
\documentclass[journal,12pt,onecolumn]{IEEEtran}
\usepackage{cite}
\usepackage{amsmath,amssymb,amsfonts,amsthm}
\usepackage{algorithmic}
\usepackage{graphicx}
\usepackage{textcomp}
\usepackage{xcolor}
\usepackage{txfonts}
\usepackage{listings}
\usepackage{enumitem}
\usepackage{mathtools}
\usepackage{gensymb}
\usepackage{comment}
\usepackage[breaklinks=true]{hyperref}
\usepackage{tkz-euclide} 
\usepackage{gvv}                                        
\def\inputGnumericTable{}                                
\usepackage[latin1]{inputenc}                                
\usepackage{color}                                            
\usepackage{array}                                            
\usepackage{longtable}                                       
\usepackage{calc}                                             
\usepackage{multirow}                                         
\usepackage{hhline}                                           
\usepackage{ifthen}                                           
\usepackage{lscape}
\usepackage{tabularx}
\usepackage{array}
\usepackage{float}


\newtheorem{theorem}{Theorem}[section]
\newtheorem{problem}{Problem}
\newtheorem{proposition}{Proposition}[section]
\newtheorem{lemma}{Lemma}[section]
\newtheorem{corollary}[theorem]{Corollary}
\newtheorem{example}{Example}[section]
\newtheorem{definition}[problem]{Definition}
\newcommand{\BEQA}{\begin{eqnarray}}
\newcommand{\EEQA}{\end{eqnarray}}
\newcommand{\define}{\stackrel{\triangle}{=}}
\theoremstyle{remark}
\newtheorem{rem}{Remark}

% Marks the beginning of the document
\begin{document}
\bibliographystyle{IEEEtran}


\title{ASSIGNMENT 1}
\author{EE24BTECH11013- DASARI MANIKANTA}
\maketitle
\newpage
\bigskip

\renewcommand{\thefigure}{\theenumi}
\renewcommand{\thetable}{\theenumi}



  \begin{enumerate}
  
	 	 \item The number of integers greater than $6,000$ that can be formed, using digits $3$,$5$,$6$,$7$ and $8$,without repetition,is\hfill{[JEE M 2015]}
    \begin{enumerate}
    \item $120$ 
    \item $72$
    \item $216$
    \item $192$ 
    \end{enumerate} 
    
	    \item If all words(with or without)having five letters,formed using the letters of the word SMALL and arranged as in a dictionary;then the position of the word SMALL is; \hfill{[JEE M 2015]}
\begin{enumerate}
    \item $ 52^{nd} $
    \item $ 58^{th} $
    \item $ 46^{th} $
    \item $ 59^{th} $
    \end{enumerate} 
 
	 \item A man $X$ has $7$ friends, $4$ of them are ladies and $3$ are men. His wife $Y$ also has $7$ friends, $3$ of them are ladies and $4$ are men. Assume $X$ and $Y$ have no common friends.Then the total number of ways in which $X$ and $Y$ together can throw a party inviting $3$ ladies and $3$ men, so that $3$ friends of each of $X$ and $Y$ are in this part$y$, is: \hfill{[JEE M 2017]}
\begin{enumerate}
\item $484$ 
\item $485$
\item $468$
\item $469$
\end{enumerate}

	\item From $6$ different novels and $3$ different dictionaries,$4$ novels and $1$ dictionary are to be selected and arranged in a row on a shelf so that  the dictionary is always in the middle. The number of such arrangements is:\hfill{[JEE M 2018]}
 \begin{enumerate}
     \item less than $500$ 
     \item at least $500$ but less than $750$
     \item at least $750$ but less than $1000$
     \item at least $1000$
     \end{enumerate}

	\item Consider a class of $5$ girls and $7$ boys. The number of different teams consisting of $2$ girls and $3$ boys that can be formed from this class,if there are two specific boys $A$ and $B$,who refuse to be members of the same team,is:\hfill{[JEE M 2019-9 Jan(M)]}
\begin{enumerate}
    \item $500$  
    \item $200$
    \item $300$
    \item $350$
    \end{enumerate}

	\item A committee of $11$ members is to be formed from $8$ males and $5$ females. If m is the number of ways the committee is formed with at least $6$ males and n is the number of ways the committee is formed with at least $3$ females, is:
    \hfill{[JEE M 2019-9April(M)]}
\begin{enumerate}
      \item $m+n=68$ 
      \item $m=n=78$
      \item $n=m-8$
      \item $m=n=68$
  \end{enumerate}  

\newpage 
\title{ Sequences and Series}
\maketitle
\section{SECTION-A}

\section{A. Fill in the Blanks}


	\item  The sum of integers from 1 to $100$ that are divisible by $2$ or $5$ is: \hfill{(1984-2 Marks)}
    
    
	    \item  The solution of the equation                                    
       ${\log_{7}\log_{5}(\sqrt{x+5}+\sqrt{x})}$ \hfill{(1986-2 Marks)}
     
     
	     \item The sum of the first $n$ terms of the series ${1^2+2.2^2+3^2+2.4^2+5^2+2.6^2+\dots}$ is
   ${n (n+1)^2 /2}$, when $n$ is even. When $n$ is odd, the sum 
   is\dots\hfill{(1988-2 Marks)}
        
          
		  \item Let the harmonic mean and geometric mean of two positive numbers be the ratio $4:5$. Then the two numbers are in 
    ratio\dots\hfill{(1992-2 Marks)}
          
     
	     \item For any odd integer $n \ge 1$, ${n^3-(n-1)^3+(-1)^{n-1} 1^3=\dots}$\hfill{(1996-1 Mark)}
       
     
	     \item  Let $p$ and $q$ be the roots of the equation                    ${x^2-2x+A=0}$ and r and s be the roots of the                     equation${x^2-18x+B=0}$.If ${p<q<r<s}$ are                                      in arithmetic progression,then A=\dots and B=\dots\hfill{(1977-2 Marks)}
     
    
    \section{ C. MCQs with One Correct Answer}
        
    
	    \item  If $x$,$y$ and $z$ are pth,qth and rth terms respectively of an A.P and also of a G.P, then ${x^{y-z} y^{z-x} z^{x-y}}$ 
        is equal to:\hfill{(1982-2 Marks)}
\begin{enumerate} 
  \item $xyz$ 
  \item $0$
  \item $1$ 
  \item none of these
  \end{enumerate}

	\item  The third term of a geometric progression is $4$. The product of five terms is\hfill{(1982-2 Marks)}
\begin{enumerate}
     \item${4^3}$ 
    \item${4^5}$
    \item${4^4}$
    \item none of these
\end{enumerate}

	\item The rational number, which equals the number$2$.${357}$ with recurring decimal is 
	\hfill{(1983-1 Mark)}
    \begin{enumerate}
        \item $\frac{2355}{1001}$\
        \item $\frac{2379}{997}$ \
        \item $\frac{2355}{999}$ \
        \item none of these 
    \end{enumerate}

\end{enumerate}
\end{document}


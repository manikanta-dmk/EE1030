\documentclass[a4paper,12pt]{article}
\usepackage{amsmath,amsfonts,amssymb}
\usepackage{graphicx}
\usepackage{geometry}
\usepackage{float} % For fixing figure positions
\geometry{a4paper, margin=1in}
\title{Lissajous Figures and Capturing One-Time Events on a CRO}
\author{ee24btech11013, ee24btech11043}

\begin{document}

\maketitle

\section*{Objective}
The purpose of this experiment is to:
\begin{enumerate}
    \item Plot at least six Lissajous figures on the Cathode Ray Oscilloscope (CRO) and explain the observed patterns using theoretical principles.
    \item Demonstrate the method to capture a one-time event on a CRO, with a detailed explanation and example.
\end{enumerate}

\section*{Introduction}
A Cathode Ray Oscilloscope (CRO) is a versatile electronic instrument used to display, analyze, and measure electrical signals. This experiment involves studying Lissajous figures, which are patterns formed when two sinusoidal signals are applied to the CRO in perpendicular directions. Additionally, the method to capture transient or one-time events on a CRO is discussed.

\section*{Lissajous Figures}

\subsection*{Theory}
Lissajous figures are the result of the superposition of two sinusoidal signals applied to the X and Y inputs of a CRO. These signals are represented as:
\begin{align*}
    x(t) &= A \sin(\omega_x t + \phi_x), \\
    y(t) &= B \sin(\omega_y t + \phi_y),
\end{align*}
where:
\begin{itemize}
    \item $A$ and $B$ are the amplitudes of the X and Y signals, respectively.
    \item $f_x$ and $f_y$ are the angular frequencies of the signals.
    \item $\phi_x$ and $\phi_y$ are their respective phase angles.
\end{itemize}
The shape of the Lissajous figure depends on the frequency ratio $f_x/f_y$ and the phase difference $\Delta \phi = \phi_x - \phi_y$. 

\subsection*{Procedure}
\begin{enumerate}
    \item Connect the X and Y inputs of the CRO to two signal generators.
    \item Set the frequencies of the signal generators to different values and adjust their amplitudes.
    \item Observe and record the patterns formed on the CRO screen.
    \item Repeat the experiment for various frequency ratios and phase differences.
\end{enumerate}

\subsection*{Results and Observations}
Six Lissajous figures were observed for the following frequency ratios and phase differences:

\begin{itemize}
    \item $f_x:f_y = 1:1$, $\Delta \phi = 0$: Straight line.
    \begin{figure}[H]
    \centering
    \includegraphics[width=0.4\textwidth]{Figure1.pdf}
    \caption{Straight line Amplitudes A:B = 1:1.}
    \label{fig:StraightLine}
    \end{figure}

    \item $f_x:f_y = 1:1$, $\Delta \phi = \pi/2$: Circle.
    \begin{figure}[H]
    \centering
    \includegraphics[width=0.4\textwidth]{Figure2.pdf}
    \caption{Circular Amplitudes A:B = 1:1.}
    \label{fig:Circle}
    \end{figure}
		\vspace{100pt}

    \item $f_x:f_y = 1:1$, $\Delta \phi = \pi/4$: Elliptical patter.
    \begin{figure}[H]
    \centering
    \includegraphics[width=0.4\textwidth]{Figure3.pdf}
    \caption{Elliptical pattern Amplitudes A:B = 2:1 (not in 1:1).}
    \label{fig:Ellipse}
    \end{figure}

    \item $f_x:f_y = 1:2$, $\Delta \phi = -\pi/2$: Two-lobed pattern.
    \begin{figure}[H]
    \centering
    \includegraphics[width=0.3\textwidth]{Figure4.pdf}
    \caption{Two-lobed Amplitudes A:B = 1:1.}
    \label{fig:TwoLobed}
    \end{figure}

    \item $f_x:f_y = 1:4$, $\Delta \phi = \pi/2$: Four-lobed pattern.
    \begin{figure}[H]
    \centering
    \includegraphics[width=0.4\textwidth]{Figure5.pdf}
    \caption{Four-lobed Amplitudes A:B = 1:1.}
    \label{fig:FourLobed}
    \end{figure}
    \vspace{100pt}
    \item $f_x:f_y = 4:3$, $\Delta \phi = \pi/2$: Complex closed figure.
    \begin{figure}[H]
    \centering
    \includegraphics[width=0.4\textwidth]{Figure6.pdf}
    \caption{Complex closed Amplitudes A:B = 1:1.}
    \label{fig:Complex}
    \end{figure}
\end{itemize}

The observed patterns are consistent with theoretical predictions.

\section*{Capturing One-Time Events on a CRO}

\subsection*{Theory}
One-time or transient events occur over a short duration and require specific techniques to be captured accurately. The CRO can be configured to trigger on specific conditions to display such events.

\subsection*{Procedure}
\begin{enumerate}
    \item Set the CRO to \textbf{Single Trigger Mode} to capture transient signals.
    \item Adjust the time base and vertical sensitivity to appropriate levels.
    \item Set the trigger level to the desired threshold.
    \item Apply the signal to the input and observe the display.
\end{enumerate}

\subsection*{Example}
Consider a pulse signal generated by a circuit during a power surge. To capture this event:
\begin{enumerate}
    \item Connect the CRO probe to the circuit output.
    \item Configure the CRO to Single Trigger Mode and set the trigger level to match the expected pulse amplitude.
    \item Start the acquisition and wait for the pulse to occur.
    \item Observe and analyze the captured waveform.
\end{enumerate}
\section*{Obervations}
Graph is generated in Y-T mode where A = 10.00 Vpp ; $f_y = 1.00kHz$; start phase = 0 ;cycles = 10
 \begin{figure}[H]
    \centering
    \includegraphics[width=0.4\textwidth]{Figure7.pdf}
    \caption{Sine function of 10 cycles.}
    \label{fig:Sin}
    \end{figure}


\section*{Conclusion}
This experiment demonstrated the generation of Lissajous figures and their dependence on frequency ratios and phase differences. Additionally, the procedure to capture transient events on a CRO was successfully implemented, highlighting its practical applications in analyzing one-time electrical phenomena.

\end{document}


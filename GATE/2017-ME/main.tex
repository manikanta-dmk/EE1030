\let\negmedspace\undefined
\let\negthickspace\undefined
\documentclass[journal]{IEEEtran}
\usepackage[a5paper, margin=10mm, onecolumn]{geometry}
%\usepackage{lmodern} % Ensure lmodern is loaded for pdflatex
\usepackage{tfrupee} % Include tfrupee package

\setlength{\headheight}{1cm} % Set the height of the header box
\setlength{\headsep}{0mm}     % Set the distance between the header box and the top of the text

\usepackage{gvv-book}
\usepackage{gvv}
\usepackage{cite}
\usepackage{amsmath,amssymb,amsfonts,amsthm}
\usepackage{algorithmic}
\usepackage{graphicx}
\usepackage{textcomp}
\usepackage{xcolor}
\usepackage{txfonts}
\usepackage{listings}
\usepackage{enumitem}
\usepackage{mathtools}
\usepackage{gensymb}
\usepackage{comment}
\usepackage[breaklinks=true]{hyperref}
\usepackage{tkz-euclide} 
\usepackage{listings}
\usepackage{gvv}
\usepackage{tikz}
\def\inputGnumericTable{}                                 
\usepackage[latin1]{inputenc}                                
\usepackage{color}                                            
\usepackage{array}                                            
\usepackage{longtable}                                       
\usepackage{calc}                                             
\usepackage{multirow}                                         
\usepackage{hhline}                                           
\usepackage{ifthen}                                           
\usepackage{lscape}
\begin{document}

\bibliographystyle{IEEEtran}
\vspace{3cm}

\title{Gate Questions}
\author{EE24BTECH11013-DASARI MANIKANTA
}
 \maketitle
% \newpage
% \bigskip
{\let\newpage\relax\maketitle}

\renewcommand{\thefigure}{\theenumi}
\renewcommand{\thetable}{\theenumi}
\setlength{\intextsep}{10pt} % Space between text and floats


\numberwithin{equation}{enumi}
\numberwithin{figure}{enumi}
\renewcommand{\thetable}{\theenumi}
\begin{enumerate}
\item The product of eigenvalues of the matrix $P$ is \\
$P = \begin{bmatrix} 2 & 0 & 1 \\ 4 & -3 & 2 \\ 0 & 2 & -1 \end{bmatrix}$\hfill[February 2017]
\begin{enumerate}
    \item $-6$
    \item $2$
    \item $6$
    \item $-2$
\end{enumerate}
\item The value of $ \lim_{x \to 0} \frac{x^3 - \sin(x)}{x} $ is\hfill[February 2017]
\begin{enumerate}
    \item $0$
    \item $3$
    \item $1$
    \item $-1$
\end{enumerate}
\item Consider the following partial differential equation for  $u(x, y)$ with the \\ constant $c > 1$.Solution of this equation is\hfill[February 2017]
\begin{center}
$\frac{\partial u}{\partial y} + c \frac{\partial u}{\partial x} = 0$
\end{center}
\begin{enumerate}
    \item$ u(x, y) = f(x + cy)$
    \item$ u(x, y) = f(x - cy) $
    \item$ u(x, y) = f(cx + y) $
    \item$ u(x, y) = f(cx - y) $
\end{enumerate}
\item The differential equation $ \frac{d^2 y}{dx^2} + 16y = 0 $ for $y(x)$ with the two boundary conditions $\frac{dy}{dx}\big|_{x=0} = 1 $ and $\frac{dy}{dx}\big|_{x=\frac{\pi}{2}} = -1 $has\hfill[February 2017]
\begin{enumerate}
    \item no solution.
    \item exactly two solutions.
    \item exactly one solution.
    \item infinitely many solutions.
\end{enumerate}
\item A six-face fair dice is rolled a large number of times. The mean value of the outcomes is \underline{\hspace{2cm}}\hfill[February 2017]
\item For steady flow of a viscous incompressible fluid through a circular pipe of constant diameter, the average velocity in the fully developed region is constant. Which one of the following statements about the average velocity in the developing region is TRUE?\hfill[February 2017]
\begin{enumerate}
    \item It increases until the flow is fully developed.
    \item It is constant and is equal to the average velocity in the fully developed region.
    \item It decreases until the flow is fully developed.
    \item It is constant but is always lower than the average velocity in the fully developed region.
\end{enumerate}
\newpage
\item Consider the two-dimensional velocity field given by 
\begin{center}
$\vec{V} = (5 + a_1 x + b_1 y) \hat{i} + (4 + a_2 x + b_2 y) \hat{j}$,
\end{center}where $ a_1, b_1, a_2 $ and  $b_2$ are constants. Which one of the following conditions needs to be satisfied for the flow to be incompressible?\hfill[February 2017]
\begin{enumerate}
    \item $a_1 + b_1 = 0$
    \item $a_1 + b_2 = 0$
    \item $a_2 + b_2 = 0$
    \item[(D)]  $a_2 + b_1 = 0$
\end{enumerate}
\item Water (density = $1000\text{kg/m}^3$) at ambient temperature flows through a horizontal pipe of uniform cross-section at the rate of $1 kg/s$. If the pressure drop across the pipe is $100 kPa$, the minimum power required to pump the water across the pipe, in watts, is\underline{\hspace{2cm}} \hfill[February 2017]
\item Which one of the following is not a rotating machine?\hfill[February 2017]
\begin{enumerate}
    \item Centrifugal pump
    \item Gear pump
    \item Jet pump
    \item Vane pump
\end{enumerate}
\item Saturated steam at 100$\circ$C condenses on the outside of a tube. Cold fluid enters the tube at 20$^\circ$C and exits at 50$^\circ$C. The value of the Log Mean Temperature Difference (LMTD) is\underline{\hspace{2cm}} $^\circ$C\hfill[February 2017]
\item The molar specific heat at constant volume of an ideal gas is equal to $2.5$ times the universal gas constant $(8.314 J/mol-K)$. When the temperature increases by $100 K$, the change in molar specific enthalpy is \underline{\hspace{2cm}} $J/mol$.\hfill[February 2017]
\item A heat pump absorbs $10 kW$ of heat from outside environment at $250 K$ while absorbing $15 kW$ of work. It delivers the heat to a room that must be kept warm at $300 K$. The Coefficient of Performance (COP) \\ of the heat pump is \underline{\hspace{2cm}}\hfill[February 2017]
\item The Poisson's ratio for a perfectly incompressible \\linear elastic material is\hfill[February 2017]
\begin{enumerate}
    \item $1$
    \item $0.5$
    \item $0$
    \item infinity
\end{enumerate}
\end{enumerate}
\end{document}

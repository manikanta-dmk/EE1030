\documentclass{article}
\usepackage{amsmath}
\usepackage{siunitx}
\usepackage{geometry}
\usepackage{listings}
\usepackage{tikz}
\usepackage{verbatim}      % For \begin{verbatim} ... \end{verbatim}
\usepackage{graphicx}      % For including images like GTKWave screenshots

\geometry{margin=1in}
\title{Digital Systems Lab Assignment-2}
\author{Murra Rajesh Kumar Reddy - EE24BTECH11043}
\date{}

\begin{document}
\maketitle

\section*{Question 1: Convolution of Two Vectors}

\subsection*{Approach}

The problem involves performing a convolution operation between two 8-element vectors of 4-bit values. The operation follows the typical convolution approach, where each element of the output vector is the result of multiplying elements of the input vectors and accumulating the results. 

The inputs are two vectors \( A \) and \( B \), each of size 8, containing 4-bit values. The output is a 16-element vector \( Y \) of 4-bit values, which represents the convolution result. This module uses nested loops to calculate the convolution, multiplying corresponding elements of \( A \) and \( B \), then adding them to form the resulting vector.

\subsection*{Functionality}

Inputs: 
  - \( A[7:0] \): 8-element vector of 4-bit values.
  - \( B[7:0] \): 8-element vector of 4-bit values.

Outputs: 
  - \( Y[15:0] \): 16-element vector of 4-bit values representing the convolution result.

Operation:
  - The convolution operation is performed by multiplying corresponding elements of \( A \) and \( B \) and then summing the results to form the output vector \( Y \).
  - The result of each multiplication is stored in a temporary register and then added to the appropriate index in the output vector \( Y \).

\subsection*{Code}

\begin{verbatim}
module convolution_A_B(
    input [3:0] A[7:0],  // 8-element vector of 4-bit values
    input [3:0] B[7:0],  // 8-element vector of 4-bit values
    output [3:0] Y[15:0] // 16-element vector of 4-bit output values
);
    integer i, j;
    reg [7:0] result;  // Temporary 8-bit register to store partial sum
    reg [3:0] temp;    // Temporary 4-bit register for storing multiplication results

    always @(*) begin
        // Initialize output Y to zero
        for (i = 0; i < 16; i = i + 1) begin
            Y[i] = 4'b0000;
        end
        
        // Perform convolution (ignoring overflow)
        for (i = 0; i < 8; i = i + 1) begin
            for (j = 0; j < 8; j = j + 1) begin
                // Multiply and accumulate the result, ignoring overflow
                temp = A[i] * B[j];  // Multiply corresponding elements (4-bit)
                result = {4'b0000, Y[i + j]} + {4'b0000, temp};  // Sum previous result and new product
                Y[i + j] = result[3:0]; // Store the 4-bit result (ignoring overflow)
            end
        end
    end
endmodule
\end{verbatim}

\subsection*{Testbench Code}

\begin{verbatim}
`timescale 1ns/1ps

module convolution_A_B_tb;

    reg [3:0] A[7:0];  // Input A: 8 elements, 4-bit each
    reg [3:0] B[7:0];  // Input B: 8 elements, 4-bit each
    wire [3:0] Y[15:0]; // Output Y: 16 elements, 4-bit each

    // Instantiate the convolution module
    convolution_A_B uut (
        .A(A),
        .B(B),
        .Y(Y)
    );

    initial begin
        // Test Case 1: Convolution with basic values
        A[0] = 4'b0001; A[1] = 4'b0010; A[2] = 4'b0011; A[3] = 4'b0100;
        A[4] = 4'b0101; A[5] = 4'b0110; A[6] = 4'b0111; A[7] = 4'b1000;

        B[0] = 4'b0001; B[1] = 4'b0010; B[2] = 4'b0011; B[3] = 4'b0100;
        B[4] = 4'b0101; B[5] = 4'b0110; B[6] = 4'b0111; B[7] = 4'b1000;

        #20;
        $finish;
    end

    // Monitor the output
    initial begin
        $monitor("At time %t, Y = %b", $time, Y);
    end

endmodule
\end{verbatim}

\subsection*{GTKWave Form}
\vspace{4cm} % Leave space for waveform image
\begin{center}
    \includegraphics[width=1\textwidth]{gtk3.png}
\end{center}
\newpage


\section*{Question 2: 8-bit Adder Design}

\subsection*{Approach}
The design implements an 8-bit adder in Verilog using a `for` loop and manual carry calculation, where carry propagates across bits. The design includes carry-in and outputs a 9-bit sum.

\subsection*{Functionality}
The adder performs the addition of two 8-bit numbers with a carry-in. The sum is stored in a 9-bit register to accommodate overflow. Intermediate carry is manually handled to simulate ripple-carry propagation.

\subsection*{Design Code}
\begin{lstlisting}[language=Verilog]
module 8_bitadder (
    input [7:0] A, B,
    input c_in,
    output reg [8:0] sum
);
    reg [8:0] c_out;

    always @(*) begin
        c_out[0] = c_in;
        for (integer i = 0; i < 8; i = i + 1) begin
            sum[i] = A[i] ^ B[i] ^ c_out[i];
            c_out[i+1] = (A[i] & B[i]) | (B[i] & c_out[i]) | (c_out[i] & A[i]);
        end
        sum[8] = c_out[8];
    end
endmodule
\end{lstlisting}

\subsection*{Testbench Code}
\begin{verbatim}
`timescale 1ns / 1ps

module tb_8_bitadder;

    // Inputs
    reg [7:0] A, B;
    reg c_in;

    // Output
    wire [8:0] sum;

    // Instantiate the Unit Under Test (UUT)
    8_bitadder uut (
        .A(A),
        .B(B),
        .c_in(c_in),
        .sum(sum)
    );

    initial begin
        // Display header
        $display("Time\tA\t\tB\t\tCin\tSum");
        $monitor("%0dns\t%b\t%b\t%b\t%b", $time, A, B, c_in, sum);

        // Test case 1: A = 5, B = 10, Cin = 0
        A = 8'd5; B = 8'd10; c_in = 0;
        #10;

        // Test case 2: A = 255, B = 1, Cin = 0
        A = 8'd255; B = 8'd1; c_in = 0;
        #10;


        // Test case 3: A = 128, B = 128, Cin = 1
        A = 8'b10000000; B = 8'b10000000; c_in = 1;
        #10;


	// Test case 4: A = 0, B = 0, Cin = 1
        A = 8'b00000000; B = 8'b00000000; c_in = 1;
        #10;


        // Test case 5: A = 170, B = 85, Cin = 1
        A = 8'b10101010; B = 8'b01010101; c_in = 1;
        #10;


        \$finish;
    end

endmodule
\end{verbatim}

\subsection*{GTKWave Simulation Output}

\begin{center}
    \includegraphics[width=0.85\textwidth]{gtk1.png}
\end{center}


\newpage

\section*{Question 3:Ripple Carry Adder Using only NAND Gates}

\subsection*{Approach}
All logic gates are implemented using only NAND gates with a defined delay. A full adder is constructed using these gates. A 4-bit ripple carry adder is then built using the full adders.

\subsection*{Functionality}
The full adder computes both the sum and carry outputs for a single-bit binary addition using logic gates. The ripple carry adder chains four full adders to compute the sum of two 4-bit numbers.

\subsection*{Design Code}
\begin{lstlisting}[language=Verilog]
// NAND Gate with 1ns delay
module nand_gate(input A, input B, output Y);
    assign #1 Y = ~(A & B);
endmodule

// XOR gate using NAND gates
module xor_gate(input A, input B, output Y);
    wire n1, n2, n3;
    nand_gate U1 (.A(A), .B(B), .Y(n1));
    nand_gate U2 (.A(A), .B(n1), .Y(n2));
    nand_gate U3 (.A(n1), .B(B), .Y(n3));
    nand_gate U4 (.A(n2), .B(n3), .Y(Y));
endmodule

// AND gate using NAND gates
module and_gate(input A, input B, output Y);
    wire n1;
    nand_gate U1 (.A(A), .B(B), .Y(n1));
    nand_gate U2 (.A(n1), .B(n1), .Y(Y));
endmodule

// OR gate using NAND gates
module or_gate(input A, input B, output Y);
    wire n1, n2;
    nand_gate U1 (.A(A), .B(A), .Y(n1));
    nand_gate U2 (.A(B), .B(B), .Y(n2));
    nand_gate U3 (.A(n1), .B(n2), .Y(Y));
endmodule

// Full Adder using gates
module full_adder(input A, input B, input Cin, output Sum, output Cout);
    wire AxorB, AandB, AxorBandCin;
    xor_gate U1 (.A(A), .B(B), .Y(AxorB));
    and_gate U2 (.A(A), .B(B), .Y(AandB));
    xor_gate U3 (.A(AxorB), .B(Cin), .Y(Sum));
    and_gate U4 (.A(AxorB), .B(Cin), .Y(AxorBandCin));
    or_gate  U5 (.A(AandB), .B(AxorBandCin), .Y(Cout));
endmodule

// 4-bit Ripple Carry Adder
module ripple_carry_adder(
    input  [3:0] A, B,
    input  Cin,
    output [4:0] Sum_ext
);
    wire [3:0] sum_internal;
    wire C1, C2, C3, Cout;
    full_adder FA0 (.A(A[0]), .B(B[0]), .Cin(Cin),  .Sum(sum_internal[0]), .Cout(C1));
    full_adder FA1 (.A(A[1]), .B(B[1]), .Cin(C1),   .Sum(sum_internal[1]), .Cout(C2));
    full_adder FA2 (.A(A[2]), .B(B[2]), .Cin(C2),   .Sum(sum_internal[2]), .Cout(C3));
    full_adder FA3 (.A(A[3]), .B(B[3]), .Cin(C3),   .Sum(sum_internal[3]), .Cout(Cout));
    assign Sum_ext = {Cout, sum_internal}; // 5-bit sum
endmodule
\end{lstlisting}

\subsection*{Testbench Code}

\begin{verbatim}
`timescale 1ns/1ps

module ripple_carry_adder_tb;

    reg  [3:0] A, B;
    reg        Cin;
    wire [4:0] Sum_ext;

    // Instantiate the ripple carry adder
    ripple_carry_adder uut (
        .A(A),
        .B(B),
        .Cin(Cin),
        .Sum_ext(Sum_ext)
    );

    initial begin
        $dumpfile("ripple_carry_adder.vcd");
        $dumpvars(0, ripple_carry_adder_tb);

        // Test Case 1: (0, 0, 0) -> 0 + 0 + 0 = 0
        A = 4'b0000; B = 4'b0000; Cin = 0; #20;

        // Test Case 2: (5, 3, 1) -> 5 + 3 + 1 = 9
        A = 4'b0101; B = 4'b0011; Cin = 1; #20;

        // Test Case 3: (15, 15, 0) -> 15 + 15 + 0 = 30
        A = 4'b1111; B = 4'b1111; Cin = 0; #20;

        // Test Case 4: (10, 13, 1) -> 10 + 13 + 1 = 24
        A = 4'b1010; B = 4'b1101; Cin = 1; #20;

        \$finish;
    end

endmodule
\end{verbatim}

\subsection*{GTKWave Result}
\vspace{4cm} % Leave space for waveform image
\begin{center}
    \includegraphics[width=1\textwidth]{gtk2.png}
\end{center}
\newpage

\section*{Delay Analysis of 4-bit Ripple Carry Adder}

\subsection*{Gate Delays}
\begin{itemize}
  \item \textbf{XOR gate}: 4 NAND gates $\Rightarrow$ delay = \SI{3}{\nano\second}
  \item \textbf{AND gate}: 2 NAND gates $\Rightarrow$ delay = \SI{2}{\nano\second}
  \item \textbf{OR gate}: 3 NAND gates $\Rightarrow$ delay = \SI{3}{\nano\second}
\end{itemize}

\subsection*{Full Adder Delay}

\textbf{Sum path:}
\[
A \oplus B = \SI{3}{\nano\second}, \quad (A \oplus B) \oplus Cin = \SI{3}{\nano\second} \Rightarrow \text{Sum delay} = \SI{6}{\nano\second}
\]

\textbf{Carry path:}
\[
A \land B = \SI{2}{\nano\second}, \quad (A \oplus B) = \SI{3}{\nano\second}, \quad (A \oplus B) \land Cin = \SI{2}{\nano\second}, \quad \text{OR} = \SI{3}{\nano\second}
\]
\[
\text{Total Cout delay} = \max(2, 3+2) + 3 = \SI{8}{\nano\second}
\]

\subsection*{Ripple Carry Delay Summary}
\begin{align*}
\text{Cout}_0 &= \SI{8}{\nano\second} \\
\text{Cout}_1 &= \SI{8}{\nano\second} + \SI{8}{\nano\second} = \SI{16}{\nano\second} \\
\text{Cout}_2 &= \SI{16}{\nano\second} + \SI{8}{\nano\second} = \SI{24}{\nano\second} \\
\text{Cout}_3 &= \SI{24}{\nano\second} + \SI{8}{\nano\second} = \SI{32}{\nano\second}
\end{align*}

\textbf{Sum Path Delays:}
\begin{align*}
\text{Sum}[0] &= \SI{6}{\nano\second} \\
\text{Sum}[1] &= \SI{8}{\nano\second} + \SI{6}{\nano\second} = \SI{14}{\nano\second} \\
\text{Sum}[2] &= \SI{16}{\nano\second} + \SI{6}{\nano\second} = \SI{22}{\nano\second} \\
\text{Sum}[3] &= \SI{24}{\nano\second} + \SI{6}{\nano\second} = \SI{30}{\nano\second}
\end{align*}

\subsection*{Final Result}
\begin{itemize}
  \item \textbf{Worst-case delay to any Sum bit:} \SI{30}{\nano\second}
  \item \textbf{Worst-case delay to final Cout:} \SI{32}{\nano\second}
\end{itemize}

\end{document}

